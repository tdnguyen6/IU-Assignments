\documentclass[12pt]{article}

% Packages here
\usepackage{geometry}
\usepackage{multicol}
\usepackage{enumitem}
\usepackage{amsmath}
\usepackage{listings}
\usepackage{xcolor}
 
% Configurations here
\definecolor{codegreen}{rgb}{0,0.6,0}
\definecolor{codegray}{rgb}{0.5,0.5,0.5}
\definecolor{codepurple}{rgb}{0.58,0,0.82}
\definecolor{backcolour}{rgb}{0.95,0.95,0.92}
 
\lstdefinestyle{mystyle}{
    backgroundcolor=\color{backcolour},   
    commentstyle=\color{codegreen},
    keywordstyle=\color{magenta},
    numberstyle=\tiny\color{codegray},
    stringstyle=\color{codepurple},
    basicstyle=\ttfamily\footnotesize,
    breakatwhitespace=false,         
    breaklines=true,                 
    captionpos=b,                    
    keepspaces=true,                 
    numbers=left,                    
    numbersep=5pt,                  
    showspaces=false,                
    showstringspaces=false,
    showtabs=false,                  
    tabsize=2
}
 
\lstset{style=mystyle}

\geometry{margin=1cm, bottom=2cm}
\setlength{\columnseprule}{1pt}

% Content here
\begin{document}
  \title{Homework 6}
  \author{Nguyen Tien Duc ITITIU18029}
  \maketitle
  \begin{multicols}{2}
    [\part*{Page 457}]
      \section*{6}
        \begin{enumerate}[label=\alph*)]
          \item \(a_n = a_{n-1}\)
          \item \(a_{n+1} = 2(n+1) = 2n + 2 = a_n + 2\) \\ \(\Rightarrow a_{n+1} = a_n + 2\)
          \item \(a_{n+1} = 2(n+1) + 3 = 2n + 3 + 2 = a_n + 2 \) \\ \(\Rightarrow a_{n+1} = a_n + 2\)
          \item \(a_{n+1} = 5^{n+1} = 5\cdot 5^n = 5a_n\) \\ \(\Rightarrow a_{n+1} = 5a_n\)
          \item \(a_{n+1} = (n+1)^2 = n^2 + 2n + 1 = a_n + 2\sqrt{a_n} + 1\)
          \item \(a_{n+1} = (n+1)^2 + n+1 = n^2 + n + 2n + 2 = a_n + 2n + 2\) \\ \(\Rightarrow a_{n+1} = a_n + 2n + 2\)
          \item \(a_{n+1} = n+1 + (-1)^{n+1} = n + 1 + (-1)^n\cdot(-1) = n - (-1)^n + 1 = n + (-1)^n - 2(-1)^n + 1 = a_n - 2(-1)^n  + 1\) \\ \(\Rightarrow a_{n+1} = a_n - 2(-1)^n  + 1 \)
          \item \(a_{n+1} = (n+1)! = (n+1)\cdot n! = (n+1)\cdot a_n\) \\ \(\Rightarrow a_{n+1} = (n+1)\cdot a_n\)          
        \end{enumerate}
      \section*{8}
        \begin{enumerate}[label=\alph*)]
          \item %a
          \begin{align*}
            a_n &= -a_{n-1} \\
             &= (-1)^2a_{n-2} \\ 
             &= \dots \\ 
             &= (-1)^n a_{n-n} \\
             &= (-1)^n a_0 \\
             &= (-1)^n\cdot5 \\
            \Rightarrow a_n &= (-1)^n\cdot5
          \end{align*}

          \item %b
          \begin{align*}
            a_n &= a_{n-1} + 3 \\ 
            &= a_{n-2}+3\cdot 2 \\
            &= a_{n-3} + 3\cdot 3 \\
            &= \dots \\
            &= a_{n-n} + 3\cdot n \\
            &= 3n + a_0 \\
            &= 3n + 1 \\ 
            \Rightarrow a_n &= 3n + 1
          \end{align*}

          \item %c
          \begin{align*}
            a_n &= a_{n-1} - n \\
            &= a_{n-2}-(n-1)-n \\
            &= a_{n-3}-(n-2)-(n-1)-n \\ 
            &= \dots \\
            &= a_{n-n}-(n+(n-1)+(n-2)\\&+(n-3)+(n-(n-1))) \\ 
            &= a_0-\frac{n(n+1)}{2} \\ 
            \Rightarrow a_ n &= \frac{-n^2-n+8}{2}
          \end{align*}
          \item %d
          \begin{align*}
            a_n &= 2a_{n-1} - 3 \\
            &= 2(2a_{n-2} - 3) - 3 \\
            &= 4a_{n-2} - 3\cdot 3 \\
            &= 4(2a_{n-3} -3) - 3\cdot 3 \\
            &= 8a_{n-3} - 7\cdot 3 \\ 
            &= \dots \\
            &= 2^n\cdot a_{n-n} - (2^n-1)\cdot 3   \\
            &= 2^na_0 - (2^n-1)\cdot 3 \\
            \Rightarrow a_n &= -2^n - (2^n-1)\cdot 3 \\
            &= -4\cdot 2^n + 3 \\
            \Rightarrow a_n &= -2^{n+2} + 3
          \end{align*}

          \item %e
            \begin{align*}
              a_n &= (n+1)a_{n-1} \\
              &= (n+1)n a_{n-2} \\
              &= (n+1)n(n-1)a_{n-3} \\
              &= (n+1)n(n-1)(n-2)a_{n-4} \\
              &= \dots \\
              &= (n+1)n\dots(n-(n-2))a_{n-n} \\
              &= (n+1)n\dots2 a_0 \\
              &= 2(n+1)!
            \end{align*}

          \item %f
            \begin{align*}
              a_n &= 2na_{n-1} \\
              &= 2n2(n-1)a_{n-2} \\
              &= 2n2(n-1)2(n-2)a_{n-3} \\
              &= 2n2(n-1)2(n-2)2(n-3)a_{n-4} \\
              &= \dots \\
              &= 2^n n\dots(n-(n-1))a_{n-n}\\
              &= 2^n n!\cdot3
            \end{align*}
          
          \item %g
            \begin{align*}
              a_n &= n-1 - a_{n-1} \\
              &= n-1 - (n-2 - a_{n-2}) \\
              &= n-1 - (n-2) + (n-3 - a_{n-3}) \\
              &= \dots \\
              &= (n-1) - (n-2) + \dots \\
              &+ (-1)^{n-1}(n-n) + (-1)^n a_{n-n} \\
              &= -1 + 2 -3 + \dots +(-1)^{n-1} n + (-1)^n7 \\
              &= \frac{2n-1+(-1)^n}{4} + (-1)^n\cdot 7
            \end{align*}
        \end{enumerate}

      \section*{12}
        In 2002: \(P_0=6.2\) Billion. \\
        \(r = 1.3\%/year = 0.013/year\) \\
        At year n after 2002:
        \begin{enumerate}[label=\alph*)]
          \item %a
          \begin{align*}
            P_1 &= P_0 + P_0\cdot r = (1.013)P_0 \\
            P_2 &= P_1 + P_1\cdot r = (1.013)P_1 \\
            \Rightarrow P_n &= (1.013)P_{n-1}
          \end{align*}

          \item %b
          \begin{align*}
            P_n &= 1.013P_{n-1} \\
            &=1.013^2P_{n-2} \\
            &=1.013^3P_{n-3} \\
            &=\dots \\
            \Rightarrow P_n &= (1.013)^n P_0 \\
            \Rightarrow P_n &= 1.013^n\cdot 6.2 \, (Billion)
          \end{align*}

          \item %c
          In 2022, \(n = 2022-2002=20\\
          \Rightarrow P_{20}=1.013^{20}\cdot 6.2= 8.0275\) (Billion) \\
          The world population will be approximately 8.0275 billion people.
        \end{enumerate}

      \section*{14}
        \begin{enumerate}[label=\alph*)]
          \item %a
            In 1999, \(S_0=\$50000\)\\
            A year late, \(S_1=S_0\cdot(1+0.05) + \$1000\)\\
            A year late, \(S_2=S_1\cdot(1+0.05) + \$1000\)\\
            Then, \(S_n=S_{n-1}\cdot(1.05) + \$1000\)

          \item %b
            In 2007,
           \(n=2007-1999=8\) \\
            Using result from c):\\
            \(S_8=1000\frac{(1.05^8-1)}{1.05-1}+50000(1.05)^8=83422\)\\
            So his salary then will be about \$83422 

          \item %c
            \begin{align*}
            S_n&=S_{n-1}\cdot(1.05) + 1000 \\
            &=(S_{n-2}\cdot(1.05) + 1000)\cdot(1.05) + 1000 \\
            &=S_{n-2}\cdot(1.05)^2+1000\cdot(1.05+1) \\
            &=(S_{n-3}\cdot(1.05)+1000)\cdot(1.05)^2 \\
            & + 1000\cdot(1.05+1) \\
            &=S_{n-3}\cdot(1.05)^3+1000\cdot(1.05^2+1.05+1) \\
            &=\dots \\
            &=S_{n-n}\cdot(1.05)^n+1000\cdot(1.05^{n-1}+1.05^{n-2} \\
            &+\dots+1.05^1+1.05^0) \\
            &=50000(1.05)^n+1000\cdot(1.05^{n-1}+1.05^{n-2} \\
            &+\dots+1.05^1+1.05^0) \\
            &=1000\frac{(1.05^n-1)}{1.05-1}+50000(1.05)^n\\
            &=70000(1.05)^n - 20000
            \end{align*}           
        \end{enumerate}
  \end{multicols}

  \begin{multicols}{2}
    [\part*{Page 471}]
    \section*{6}
      Let \(a_n\) be the number of messages \\
      transmitted in n \(\mu s\). \\
      One signal need 1 \(\mu s: a_{n-1}\)\\
      One signal need 2 \(\mu s: a_{n-2}\)\\
      One signal need 2 \(\mu s: a_{n-2}\)\\\\
      For \(n\geq2: a_n=a_{n-1}+2a_{n-2}\)\\

      In 0 \(\mu s\), only empty message can be sent.
      \begin{center}
        \(\Rightarrow a_0=1\)
      \end{center}

      In 1 \(\mu s\), only 1 message can be sent.
      \begin{center}
        \(\Rightarrow a_1=1\)\\
        Using characteristic equation:\\
        Let \(a_n=r^2,a_{n-1}=r, a{n-2}=1\)\\
        \(\Rightarrow r=2\) or \(r=-1\)\\
        Solution recurrence relation: \\
        \(a_n=\alpha_1r^n_1+\alpha_2r^n_2\)
      \end{center}

      \begin{align*}
        \text{From initial conditions, we have:}\\
        1 &= a_0=\alpha_1+\alpha_2\\
        1 &= a_1=2\alpha_1-\alpha_2\\
        \text{Then:}\\
        \alpha_1&=\frac{2}{3}\\
        \alpha_2&=\frac{1}{3}
      \end{align*}
      The solution to the recurrence relation is:\\
      \[a_n=\frac{2}{3}2^n+\frac{1}{3}(-1)^n\]
    \section*{8}
      \begin{enumerate}[label=\alph*)]
        \item The recurrence relation is simply:\\
        \[L_n=\frac{1}{2}(L_{n-1}+L_{n-2})\]
        \item The characteristic equation:\\
        \[r^2-\frac{1}{2}r-\frac{1}{2}=0\\
        \Rightarrow r=-\frac{1}{2} \text{ or } r=1\]\\
        The general solution is:\\
        \[a_n=\left(-\frac{1}{2}\right)^n\alpha_1+1^n\alpha_2\]\\
        We have:\\
         \[L_1=100000=\frac{1}{2}\alpha_1+\alpha_2\]\\
         \[L_2=300000=\left(\frac{1}{2}\right)^3\alpha_1+\alpha_2\]\\
         Then:
         \[\left\{\alpha_1=\frac{8000000}{3}\text{ and }
         \alpha_2=\frac{700000}{3}\right\}\]
         Then:
         \[L_n=\left(\frac{800000}{3}\right)\left(\frac{-1}{2}\right)^n+\left(\frac{700000}{3}\right)\]
      \end{enumerate}
    \section*{10}
      If \(r_0\) is the only root of \(r^2-c_1r-c_2=0\)\\
      % Assume: \(a_n=\alpha_1r_0^n+\alpha_2nr_0^n\)\\
      Then: \(r^2-c_1r-c_2=(r-r_0)^2=0\\
      \Leftrightarrow r^2-c_1r-c_2=r^2-2rr_0+r_0^2\\
      \Leftrightarrow c_2=-r_0^2 \text{ and } c_1=2r_0\)\\
      Prove the solution:
      \begin{align*}
        c_1a_{n-1}+c_2a_{n-2}&=2r_0a_{n-1}-r_0^2a_{n-2}\\
        &=2r_0(\alpha_1r_0^{n-1}+\alpha_2nr_0^{n-1})\\
        &-r^2_0(\alpha_1r_0^{n-2}+\alpha_2nr_0^{n-2})\\
        &=\alpha_1r^n_0+\alpha_2nr^n_0\\
        &=a_n
      \end{align*}
      Thus the sequence {\(a_n\)} is a solution of the recurrence relation.

    \section*{12}
      We have for \(n\geq3\)
        \[a_n=2a_{n-1}+a_{n-2}-2a_{n-3}\]\\
      The characteristic equation is:
        \[r^3-2r^2-r+2=0\Rightarrow r=-1;r=1;r=2\]
      Then the solution is:
        \[a_n=(-1)^n\alpha_1+\alpha_2+2^n\alpha_3\]
      We also have:
        \[a_0=3=\alpha_1+\alpha_2+\alpha_3\]
        \[a_1=6=-\alpha_1+\alpha_2+2\alpha_3\]
        \[a_2=0=\alpha_1+\alpha_2+4\alpha_3\]
      Then: \(\alpha_0=-2; \alpha_1=6; \alpha_2=-1\)
      The solution is:
      \[a_n = -2\cdot(-1)^n + 6 - 2^n\]
  \end{multicols}

  \part*{HanoiTower.cpp}
    \lstinputlisting[language=c++]{TowerOfHanoi.cpp}

\end{document}
